\section{Version Control Systems}
\label{sec:vcs}

Version control systems (VCS) are an essential part of a variety of
data oriented applications. The ability to access arbitrary revisions of
some data and the implicit dependency tracking across sets of data
allow these applications to backtrack to a specific point in time and
examine that state. This becomes invaluable when one desires to revert to
a previous state or examine how a particular part of data has evolved.

% also sometimes called source code management or source configuration
% management (SCM) systems,

There is no agreed upon terminology across the many different
VCS. Depending on the backgrounds of their respective creators, they
have each tried to define the system in their own terms. However, a
few basic concepts exist across the many different systems.  By
outlining the basic concepts of version control, it is the intention
that the reader can apply any previous knowledge about VCS to this
rather generalized model and that it will provide a gentle
introduction for readers with no prior knowledge about VCS.

\subsection{Versioning Data}

The primary reasons for using a VCS is the ability to track various
types of data and to be able to access different versions of it.
Apart from the content of files, versioning can be applied to other
types of data, such as mode flags and file and directory names,
allowing rename tracking and checkouts with the correct permissions.
Some use-cases may also require other data associated with the VCS to
be versioned, data which is normally not under version control. This
can be the case for alternative usage modes for the VCS.  For example,
a patch queue manager may be implemented, using the existing VCS
infrastructure--particularly, the backing store--but presenting a view
of the repository as a set of patches as opposed to a linear history.
However whereas a VCS is typically built around the concept of history
being immutable, the patch queue manager wants to be able to reorder
patches as it sees fit. This might motivate that the history of a
patch or patch series be recorded in the VCS, something that is not
implicitly supported by the VCS but is built on top of it. In
conclusion, there are a lot of different types of data which can be
versioned, some of which is critical and some may only be under
version control for convenience.

At a glance, a VCS functions much like a database management system
(DBMS), and the work flow associated with using a VCS is very
similar to the transaction-based updating of databases. In a DBMS, all
the tables and field that need to be updated are changed by executing
a series of commands, and a finalizing commit is then issued to have
the changes mirrored in the database. This resembles how a developer
``locally'' changes one or more files over a series of edits and
finally asks the VCS to integrate the changes into its current state.
However, whereas most DBMS only keep a journal of recent changes until
some persistent on-disk state has been ensured, the VCS will keep
appending all changes to its journal.

An important point in the above example is the atomicity of the
operation that updates the VCS with new changes. This ensures that the
VCS always has a consistent and well-defined view of the development
history. Guaranteeing atomicity is especially important in
environments where a repository is shared and several developers can
simultaneously try to update it. To guard against such race
conditions, locking of either the full repository or only the files being
updated is often used. For big repositories with many developers,
the locking must be as fine-grained as possible, so that lock contention
is reduced.

% Versioning by snapshots and tracking relation
% -> versus versioning by changesets?
% Versioning files versus versioning directories.

In the following, a snapshot oriented approach to data-versioning is
assumed. Other approaches exist, such as changeset-oriented
versioning, where changes are expressed in terms of deltas instead of
a succession of snapshots. One of the benefits of changesets is that
they more densely express how the data being versioned changes between
sets of states and they are more similar to how developers use and
think of a VCS. Since changesets are only deltas, deriving a snapshot
from a list of changeset may require that several deltas are examined.
Thus with regard to representation, the snapshot-oriented approach is
simpler and it is how git versions data.

% No dates
% 
% Dates can only be used as a heuristic for commit relations
% 
% It's not really a problem. The timestamps don't ``matter''. The only thing
% git really cares about is the dependency order.
% 
% If commit timestamps aren't ordered, some of our programs may look at
% unnecessarily many commits because the heuristics use the (committer)
% timestamp as a way to guess which leg of a parenthood chain to take, but
% that's just a guess.

% Allthough a VCS can be used for keeping

% Another important reason is the desire to consistent / backup

\subsection{Collaboration and Snapshots}

\newlength{\RevWidth}
\newlength{\BranchWidth}
\settowidth{\RevWidth}{\scriptsize{\bf A}}
\settowidth{\BranchWidth}{\scriptsize{\bf 0.2-prerelease}}
\newcommand{\Rev}[1]{[name=#1]\makebox[\RevWidth]{\scriptsize{#1}}}
\newcommand{\RevNode}[1]{[name=#1]\makebox[\RevWidth]{\scriptsize{}}}
\newcommand{\TextNode}[2]{[name=#1,linecolor=white]\makebox[\RevWidth]{\scriptsize{#2}}}
\newcommand{\Branch}[2]{[name=#1,linewidth=2pt]\makebox[\RevWidth]{\scriptsize{#1}} & [name=Z,linecolor=white]\makebox[\RevWidth]{\scriptsize{#2}}}
\newcommand{\MergeRev}[1]{[name=#1]\makebox[\RevWidth]{\scriptsize{#1}}}
\newcommand{\RootRev}[1]{[name=#1,linestyle=dotted]\makebox[\RevWidth]{\scriptsize{#1}}}
\newcommand{\RevLine}[2]{\ncline{#1}{#2}}

% fillcolor=lightgray,fillstyle=solid

\begin{figure}%:[!hbp]
\begin{center}

	\begin{psmatrix}[colsep=1,rowsep=0.5,mnode=circle]
		\RootRev{A} & \Rev{B} & \Rev{C} & \Rev{D} & \Rev{E} & \Branch{F}{0.2-prerelease} \\
		            &         &         & \Rev{G} & \Rev{H} & \Branch{I}{0.1-release}
		% Node connections
		\psset{nodesep=1pt}
		\RevLine{A}{B}
		\RevLine{B}{C}
		\RevLine{C}{D}
		\RevLine{D}{E}
		\RevLine{E}{F}
		\RevLine{C}{G}
		\RevLine{G}{H}
		\RevLine{H}{I}
	\end{psmatrix}

	\caption{Parallel lines of development. The dotted circle is
	the initial state and branch heads are marked by a thick solid
	circle.}

	\label{fig:parallel-dev}

\end{center}
\end{figure}

As the source code being managed using the VCS gradually matures,
snapshots of the source code will be released. Each of these releases
must be maintained while the main development line enters the
prerelease stage in order to prepare the next release. This makes it
necessary to be able to simultaneously track multiple lines of
development and exchange bug fixes and other changes between them.
This common scenario illustrates the importance of both of snapshot
versioning and support for collaboration. With multiple lines of
development, developers can work in parallel on different snapshots
and each feed the changes back to the VCS. For example, one team can
work on stabilizing the release by fixing bugs, while another team can
work on adding new features to the line of development where the next
release is prepared. This scenario is depicted in Figure
\ref{fig:parallel-dev}.  The VCS considers the 0.1-release snapshot
and the 0.2-prerelease snapshot to be completely separate in terms of
development. However, because they share a common parent snapshot,
namely the snapshot on which the release was originally based, the VCS
is able to find the changes made to the release snapshot and integrate
them in the prerelease snapshot without overwriting changes made in
the prerelease snapshot.

The basic entity in the above is the concept of a snapshot, a specific
state, which in this context refers to a set of files ordered in some
directory tree. Snapshot versioning builds on top of this by also
recording dependencies between snapshots, how a snapshot relates to
other snapshots, such as which snapshot is the parent of a snapshot.
By singling out a specific subdirectory or file over a series
of versioned snapshots, the complete history of one particular module
of interest can be viewed and changes can be attributed to the
appropriate developers. This is important for the review process when
collaborating, since it allows a developer to query who added a file,
implemented a feature, or introduced a bug.

In terms of collaboration, another important aspect is the
distribution of snapshots, and the seamless integration of changes
between different source trees. A VCS makes it trivial to update a
source tree to the most recent version, taking into consideration any
local changes. This distribution aspect is not only useful for
developers, but also to beta users who help out testing unreleased
features.

% For a decentralized VCS the distribution aspect becomes even
% more important, since development can happen in completely different
% trees and synchronization is the only method of integrating changes
% across them.

\subsection{Centralized versus Decentralized Model}

% What limitation / constraints does decentralization introduce?
% 
%  - Security: Hash-functions and security.
% 
%  - All or nothing: scalable synchronization of repositories.

Historically, the world of version control has been dominated by
centralized systems, such as CVS, where clients connect to an
authoritative server providing access to a single shared data
repository. The centralized model, as implemented by existing systems,
has encouraged development to happen in a linear fashion instead of in
parallel. The primary reason is that branching has been an expensive
operation, which has limited its usage to only big feature branches.
Being limited to mostly linear development puts certain restrictions
on how collaboration can happen and is one of the reasons the
centralized model works best in an environment where developers are
working close together and are able to coordinate major changes. 

Having an authoritative end point simplifies certain parts of the
system, such as time stamping, and ensures that every part of the
system shares the same view of the development history. The latter is
important when it comes to ordering changes and maintaining version
numbers on revisions or files, and is one of the reasons why there cannot be any stable version numbers in a
decentralized\footnote{\emph{Decentralized} is used in the following
instead of \emph{distributed} to prevent confusion with the
distribution aspect of VCS.} VCS since there simply is no way to
assign meaningful version numbers across several logically distinct
end points without them having to continuously negotiate with all other
end points. Instead many decentralized VCS uses a revision ID to
identify snapshots.

\begin{figure}%[!hbp]
\begin{center}
	\begin{psmatrix}[colsep=1,rowsep=1,mnode=circle]
		             &             &             & \RevNode{A}  &	       &  		 &   & &  \TextNode{I}{Linus's kernel repository} \\
		             & \RevNode{B} &             &  \RevNode{C} &              & \RevNode{D}     &   & &  \TextNode{J}{Subsystem maintainer repositories} \\
	        \RevNode{E}  & \RevNode{S}  & \RevNode{F} & \RevNode{T} & \RevNode{G}  &  \RevNode{U}    &  \RevNode{H}  & & \TextNode{K}{Other developers and testers} 
		% Node connections
		\psset{nodesep=1pt}
		\RevLine{A}{B}
		\RevLine{A}{C}
		\RevLine{A}{D}
		\RevLine{A}{G}
		\RevLine{B}{E}
		\RevLine{B}{F}
		\RevLine{C}{F}

		\RevLine{D}{H}
		\RevLine{D}{U}
		\RevLine{C}{G}
		\RevLine{C}{T}
		\RevLine{B}{S}
	\end{psmatrix}

	\caption{A simplified illustration of the hierarchic structure
	of repositories for the Linux kernel project. Linus primarily
	integrates changes from the repositories of the subsystem
	maintainers, which themselves have changes contributed to them
	by other developers. This hierarchy ensures that the project
	can scale to many participating developers, because changes
	consequently go through a series of developers so that no
	one developer is responsible for integrating everything.}

	\label{fig:hierarchic-dev}

\end{center}
\end{figure}

While there may not be any authoritative end point per se in a
decentralized VCS, end points or repositories may be ordered in some
hierarchical way. For example, one repository can be an authoritative
upstream from where all other repositories fetch changes. Such an
elaborate structure has evolved around the Linux kernel tree, where
Linus Torvalds maintains an authoritative repository on which other
developers base their work. This arrangement, illustrated in Figure
\ref{fig:hierarchic-dev}, is, however, only controlled by policy
practiced by developers, it is not controlled down in the VCS.

The lack of an implicit authoritative end point consequently introduces the
requirement that the development history be replicated on all end
points. Being able to locally access and modify the complete replica
of the development history allows development to happen in a more
disconnected fashion than is supported by a centralized VCS. A company
can fork\footnote{Start a project from an existing code base.} an
external code base and internally work on a new feature without
being required to integrate changes back until they are ready and still be
able to integrate changes from the original end point if required.
This flexibility means that a decentralized VCS supports a wider range
of development flows, including the one supplied by centralized VCS,
since the centralized model is merely a special case of
the decentralized model.

Compared to the centralized model, the decentralized model raises a
lot of issues. The lack of global time means that timestamps can only
be used as heuristics of  when something occurred.  The same may also
apply to author attribution if there is no way to verify the identity
of a developer. Another issue is with name spaces across different
repositories where clashes can occur unless a naming policy is
implemented. As a last example, changes must to some extent be ordered
in a well-defined way and globally identifiable across every
replication so that a consistent view of the development history
prevails. Each of these issues must be addressed, often requiring a
compromise between conflicting goals.

Since git uses a decentralized model, the rest of the walk-through will be focused
on decentralized version control systems.

\subsection{The Repository}

The most important concept of version control is the notion of the
\emph{repository}\footnote{Also sometimes referred to as an
\emph{archive}.}.  It is in the repository that the VCS stores the
state information about the files and content it is tracking. Apart
from containing all the different versions of the content, the
repository is also used for storing various meta-data, such as which
files are currently being tracked.

% Update semantics
%To increase data integrity
As can be expected, much of the updating of a repository happens in an
append-only fashion. That is, every state that enters the repository
immediately becomes read-only, since keeping every previous state
around is what allows the VCS to backtrack. Only a few parts
of the repository are not version controlled in this respect, namely
the parts responsible of keeping track of which version is the most
recent; they will continuously be modified.

In a decentralized VCS, the repository is usually assumed to be
self-contained and will be replicated at each end
node. However, a repository may hold several different lines of
development, and in some circumstances, the replication may only be
partial, depending on what part of the repository the user doing
the replication is interested in. Partial replication, where only a
subset of a remote repository's available lines of development is
replicated, is perfectly well-defined, and the end result itself will be
a self-contained repository.

% the result is -> it leads to

% Theis stems from the fact that a repository may hold several different
% lines of development.

\begin{comment}

\subsection{Storage Model}

Two Different Approaches to Versioning

To give an idea of how a VCS works, two different models or views will first
be investigated:

In the following two different views of how a VCS operates will be
investigated:

\begin{description}

\item[Database]
%
	The VCS is a database which is used to \emph{track content}.
	It is data-oriented in that you can have different
	blobs of data each with a unique ID. which may and may not be related.

\item[Virtual file system]
%
	A VCS can be thought of as providing a virtual file system layer.
	
	Snapshots of a directory tree are stored, which enables previous
	versions of files and directories to be examined and accessed. Apart
	from being used as a mean to backup and distribute snapshots, this
	also allows one to maintain 

	It is a very file-oriented approach in that the interface provided by the VCS enables
	files to be added, removed, and moved around in the directory tree.  where the VCS provides a
	virtual file system.  It is possible to add and remove files.
	Move them between different directories.

	There are two views %

Views: a virtual file system, a database

\end{description}

\end{comment}

\subsection{Branches and Tags}

\begin{figure}%[!hbp]
\begin{center}

	\begin{psmatrix}[colsep=1,rowsep=0.5,mnode=circle]
		            &         &         & \Rev{L} & \Branch{M}{Feature branch} \\
		\RootRev{A} & \Rev{B} & \Rev{C} & \Rev{D} & \Rev{E} & \MergeRev{F}  & \Branch{G}{Trunk} \\
		            & \Rev{H} & \Rev{I} & \Rev{J} & \Branch{K}{0.1-release}
		% Node connections
		\psset{nodesep=1pt}
		\RevLine{A}{B}
		\RevLine{B}{C}
		\RevLine{C}{D}
		\RevLine{D}{E}
		\RevLine{E}{F}
		\RevLine{F}{G}
		\RevLine{C}{L}
		\RevLine{L}{M}
		\RevLine{M}{F}
		\RevLine{A}{H}
		\RevLine{H}{I}
		\RevLine{I}{J}
		\RevLine{J}{K}
		\RevLine{I}{J}
	\end{psmatrix}

	\caption{Parallel lines of development in different branches.
	The dotted circle is the initial state and branch heads are
	marked by a thick solid circle. The trunk and release branch
	is completely separate.}

	\label{fig:branches}

\end{center}
\end{figure}

Different lines of development are also known as \emph{branches}.
Each branch has a \emph{branch head} that refers to the most recent
snapshot in the branch. The branch head of the main development branch
is usually referred to as the \emph{trunk}. In the following, branch
and branch head will be used interchangeably.

Branching is mostly used when releasing or preparing a release of a
stable version. This enables developers to ``freeze'' development in
the stable version while still being able to apply fixes to the stable
branch and continue developing in the trunk. Branching can also be
used in topic branches to group changes together, and
to keep experimental feature development from interfering with other
changes. This is illustrated in Figure \ref{fig:branches}.

Although different branches are somewhat separate entities, it is
possible to synchronize changes between them. This is especially
useful when a fix has been applied in a release branch and it should
also propagate to the trunk. The methods used are ``cherry
picking,'' where only a subset of the changes in one branch are
applied, and merging, described in Section \ref{sec:merging}, where all
changes are applied.

Whereas a branch is something in flux, a \emph{tag} is a static reference to
a specific point of development; a method of bookmarking important
milestones in the project. Tags are usually created in the event of a
release, to specifically record the state of the content being
tracked, so it is possible to easily refer to that state, e.g. when
creating an archive for distribution. In some VCS, tags can be signed
with a cryptographic key in order to introduce a certain level of
trust that it is this and only this version that should be considered
valid. The ability to verify and trust tags is especially important in
a decentralized VCS, where development can happen across a series of
both trusted and untrusted repositories.

For both branches and tags, private copies local to the repository and
public copies exported or shared with a remote repository can exist.
Consequently, it is important to be aware of possible name space
collisions across different repositories. In a hierarchical relation
between two end points, where one repository can be regarded as the
\emph{origin} or \emph{upstream}, the downstream repository should
follow the naming convention dictated by its upstream. However, in
systems where there is no hierarchy, the need can arise for the various
contributors to agree on and impose a naming policy to avoid collisions.

\subsection{Working Trees and Working Bases}

As noted, a repository may contain many different snapshots of the content being
tracked by the VCS. Since it is not possible to work directly on the
different repository states, a repository is usually accompanied by a
\emph{working tree}\footnote{Also sometimes known as a \emph{working
copy}.}, which is a mutable copy of the content of a snapshot. In other words, the working tree
is a \emph{check out} of a specific snapshot contained in the repository.
Although it may be any snapshot, a working tree is commonly a check out
of a branch head, since the most recent version of the branch is
usually the most interesting one.  However, in the act of bug hunting for example,
the working tree can be any recorded state.

The basic development work flow includes checking out a specific branch
head and making changes to one or more files. Apart from content
changes, modifications of the working tree can also include file mode
changes, for example, making a file executable.  In this process,
it is useful to be able to see the difference between the current
working tree and the initial checkout, sometimes called the
\emph{working base}. Changes that should not be kept can be reverted
using the working base, by simply replacing parts of the working tree
with those in the working base.

The existence of a working tree is not a requirement unless access to
files from repository snapshots is necessary. In fact, public
accessible repositories solely used for distribution purposes are
usually \emph{bare} in the sense that they do not have any working
tree attached. There is simply no point in having a working tree for a
repository only used for synchronizing updates.

\subsection{Commits and Revisions}

% Atomic commits?

When changes to the working tree have been tested and verified to be
correct, they should be \emph{checked into} the repository, or more
precisely, the modified working tree should be recorded as the new state. This
process is called \emph{committing} and will result in a new
\emph{revision} being created from the current working tree. In this
way, there will exist a revision for each state recorded in the
repository and each revision will have a unique ID that can be used
for addressing a specific state.

Apart from simply recording content changes, the commit will also hold
information such as the date of the change and commit and the author and
the committer of the change. Usually, it will also be accompanied by a
log message which can be used for describing the change, e.g. justify
why the change is needed, and for bug fixes, what previous revision
the commit amends.

Finally, the commit operation is also responsible for updating the
head of the branch that is currently checked out, so that it points to
the new revision ID. To ensure atomic commits, this must be done using
locking and only once it has been ensured that the revision snapshot
has been written onto the backing store.

\subsection{Merging}
\label{sec:merging}

\begin{figure}%[!hbp]
\begin{center}

	\begin{psmatrix}[colsep=1,rowsep=0.5,mnode=circle]
		           && &         & \RootRev{P}   & \Rev{Q} & \Branch{R}{Feature branch} \\
		\RootRev{A} & \Rev{B} & \Rev{C} & \Rev{D} & \Rev{E} & \Rev{F}  & \Rev{G}  & \Branch{H}{Trunk} \\
		            &         & \Rev{I} & \Rev{J} & \Rev{K} & \Rev{L} & \Branch{M}{0.1-release} \\
		            &         &         &         & \Rev{N} & \Branch{O}{Bug fix branch}
		% Node connections
		\psset{nodesep=1pt}
		\RevLine{A}{B}
		\RevLine{B}{C}
		\RevLine{C}{D}
		\RevLine{D}{E}
		\RevLine{E}{F}
		\RevLine{F}{G}
		\RevLine{G}{H}
		\RevLine{K}{F}

		\RevLine{P}{Q}
		\RevLine{Q}{R}
		\RevLine{R}{H}

		\RevLine{B}{I}
		\RevLine{I}{J}
		\RevLine{J}{K}
		\RevLine{K}{L}
		\RevLine{L}{M}
		\RevLine{J}{N}
		\RevLine{N}{O}
		\RevLine{O}{M}
		\RevLine{M}{H}
	\end{psmatrix}

	\caption{A complex revision graph with several merge
	revisions. For the merge in revision M, there is only one
	merge base, namely revision J. The merge in revision H merges
	two different branches and consequently has to find a merge
	bases for both revisions being merged with revision G.  The
	branch with revision G and the branch with revision M
	intersect at two points: revision B, just before the initial
	divergence; and later, revision K, where the release branch
	was merged into the trunk.	Finally, no merge base exists
	between revision G and revision R when they are merged
	together in revision H.}

	\label{fig:merge-base}

\end{center}
\end{figure}

Where branching enables a line of development to be split into several
different and completely separate entities, \emph{merging} provides a
way to join one or more branches so the resulting branch will contain
all the content changes made across the branches being merged. At the
lower level, merging is actually performed on revisions, so the result
of a merge is in some ways similar to the commit operation mentioned
above in that a new revision will be created; however, the difference is
that a merge revision will have two or more parent revisions.

Since each branch of the set that is being joined together can potentially have renamed,
moved, or modified the same content, merging can be a very complex
operation, where several revisions must be examined, and where
human-intervention may be required to resolve conflicting content
changes. Although several branches can be merged in one
batch\footnote{A so-called ``octopus'' merge.}, the branches are
usually merged one at a time, reducing the problem to just merging two
revisions.

% Merging is always local.
While merging can be used to join any two branches regardless of the
origin of each--remote repository or the local one--merging must
always happen locally with the entire ancestry graph of revisions
being available of each branch being merged. Therefore, to merge a remote
branch, it must first be fetched into the local repository before the
merging can begin.  Again, this reduces the problem to just merging
two local revisions.


When merging two revisions, the first step is to find a \emph{merge
base}, which is a ancestor revision shared by the two branches
being merged. If development of some feature has happened in a
separate branch which was branched off from trunk, the merge base when
merging this feature branch into the trunk will simply be the branch
point of the feature branch. However, in some circumstances finding
the merge base can be more difficult. If, for example, the branches
being merged are themselves ``derivatives'' of other branches, there
might be multiple merge bases from which to choose, and the best merge base
might only exist a long way back in the history. Also note that in
some circumstances, no merge base may even exist. This can happen if
two branches that are completely unrelated, in that they do not
originate from the same code base and do not share any history, are
merged. A few examples are depicted in Figure \ref{fig:merge-base}.

Comparing against the merge base, the directory trees of the two
revisions are examined in order to find out what has been changed in
either revision. If a file has been modified in one revision but not
in the other, the modified version is accepted into the new merged
revision.  If, however, a file has been modified by both revisions, a
file-level merge needs to happen. As long as different parts of the
file were modified this will succeed, but if some of the changes in
the two revisions intersect, a conflict will be generated.

\begin{figure}%[!hbp]
\begin{center}

	\begin{psmatrix}[colsep=1,rowsep=0.5,mnode=circle]
		\RootRev{A} & \Rev{B} & \Rev{C} & \Branch{D}{Trunk} \\
		            &         &         &         &  \Rev{E} & \Rev{F} & \Branch{G}{Feature branch}
		% Node connections
		\psset{nodesep=1pt}
		\RevLine{A}{B}
		\RevLine{B}{C}
		\RevLine{C}{D}
		\RevLine{D}{E}
		\RevLine{E}{F}
		\RevLine{F}{G}
	\end{psmatrix}

	\caption{The changes in the trunk are a subset of the changes
	in the feature branch, so it fast-forwards when merging with
	the feature branch.}

	\label{fig:fast-forward}

\end{center}
\end{figure}

% When <current> is a descendant of <merged>, or <current> is an
% -ancestor of <merged>, no new commit is created and the <message>
% -is ignored.  The former is informally called ``already up to
% -date'', and the latter is often called ``fast forward''.

When integrating changes, the trivial case is when changes in a branch
are a subset of the changes in the branch being merged. In terms of
merge bases, this case is identified by the merge base of the two
branches being the head of the branch being updated. Merging will
reduce to simply \emph{fast-forwarding} this branch so it ends up
being identical to the branch being merged from. Figure
\ref{fig:fast-forward} shows an example where a feature branch is
being used for experimental development but is based on the trunk.  If
the feature branch is merged into the trunk it will simply
fast-forward the trunk.

An important feature when merging is rename tracking enabling a change
made to a file in one revision to appear in the merged revision
even if the other revision renamed the file. Rename tracking, however,
also adds several corner cases, e.g. a file could have been renamed in both
revisions or, in a more obscure example, could have been renamed such that a
path that pointed to a file in the merge base now points to a
directory. Although these corner-cases may not be very common, a robust VCS must
still detect and handle them.

Before a merge with conflicting changes can be completed, all
conflicts need to be \emph{resolved}. The working tree is often used
as a staging area for this, where the VCS can add conflict markers to
the files with conflicts so they can easily be edited in order to
resolve the conflict.

% While most merges can be performed automatically without problems, the
% possibility of a merge conflict is the main cause of the merge
% operation being one of the only operations that requires human
% supervision.

\subsection{Pushing and Pulling Changes between Repositories}

\begin{figure}%[!hbp]
\begin{center}

	\newlength{\RepoWidth}
	\settowidth{\RepoWidth}{\scriptsize{\bf Master repository}}
	\newcommand{\Repo}[1]{[name=#1]\psframebox[cornersize=absolute,linearc=3pt]{\makebox[\RepoWidth]{\scriptsize{\bf Repository #1}}}}
	\newcommand{\MRepo}[1]{[name=#1]\psframebox[cornersize=absolute,linearc=3pt]{\makebox[\RepoWidth]{\scriptsize{\bf Master repository}}}}
	\newcommand{\RepoLine}[3]{\ncline{#1}{#2}\naput{\scriptsize{\emph{#3}}}}

	\begin{psmatrix}[colsep=2,rowsep=2,mnode=r]
		         & \Repo{A} \\
		\Repo{B} & \MRepo{E} & \Repo{C} \\
		         & \Repo{D}
		% Node connections
		\psset{nodesep=5pt,arrows=->}
		\RepoLine{A}{B}{}
		\RepoLine{B}{A}{RW}
		\RepoLine{B}{E}{RW}
		\RepoLine{E}{C}{RO}
		\RepoLine{B}{D}{RO}
		\RepoLine{E}{D}{RO}
		\RepoLine{E}{A}{RO}
	\end{psmatrix}

	\caption{Changes can be pulled and pushed asynchronously
	between repositories. They may propagate through both
	read-only (RO) repositories and writable (RW) repositories,
	that both allow pulling and pushing changes.}

	\label{fig:pull-push}

\end{center}
\end{figure}

While repositories may be synchronized in a synchronous manner, e.g.
if one is maintained as an exact mirror of the other, it is common to
distinguish between \emph{pushing} local changes to a remote
repository and \emph{pulling} changes from a remote repository into a
local one. This separation has the property of making the pull
operation completely read-only with respect to the repository being
synchronized from, a very common scenario, since a VCS is also used for
distributing changes, and often lots of people will only be
``consumers'' of updates. As already mentioned, all repositories in a
decentralized VCS are a self-contained replica. Changes can therefore
be synchronized in an asynchronous way and can propagate to a
repository through a variety of paths, as shown in Figure
\ref{fig:pull-push}.

% The separation also underlines that often
% changes trickle only in one direction, the asynchronous workflow where
% you continously rebase your local work on changes pulled from the
% remote repository until the changes can be integrated.

Since an existing repository is required for pushing and pulling,
the initial pull from a repository is therefore
special and often done in conjunction with initializing an empty local
repository to pull changes into. This initial replication process is
referred to as \emph{cloning}, since it will result in an
identical repository being created, and may involve downloading a
considerable amount of data. Cloning will also take care of mapping
remote branches to local branches which are later used when
synchronizing changes with the remote repository.  Apart from setting
up the repository, it also includes checking out a working tree. After
cloning, the new repository may itself be cloned, which will result in
yet another copy of the repository.

\begin{figure}%[!hbp]
\begin{center}

	\begin{psmatrix}[colsep=1,rowsep=0.5,mnode=circle]
		            &         &         &  \Rev{D} & \Rev{E} & \Rev{F}  & \Branch{G}{Local branch} \\
		\RootRev{A} & \Rev{B} & \Rev{C} \\
		            &         &         &  \Rev{I} & \Rev{J} & \Branch{K}{Remote branch}
		% Node connections
		\psset{nodesep=1pt}
		\RevLine{A}{B}
		\RevLine{B}{C}
		\RevLine{C}{D}

		\RevLine{D}{E}
		\RevLine{E}{F}
		\RevLine{F}{G}

		\RevLine{C}{I}
		\RevLine{I}{J}
		\RevLine{J}{K}
	\end{psmatrix}

	\begin{psmatrix}[colsep=1,rowsep=0.5,mnode=circle]
		            &         &         &  \Rev{D} & \Rev{E} & \Rev{F}  & \Rev{G}  &  \Branch{H}{Local branch} \\
		\RootRev{A} & \Rev{B} & \Rev{C} \\
		            &         &         &  \Rev{I} & \Rev{J} & \Branch{K}{Remote branch}
		% Node connections
		\psset{nodesep=1pt}
		\RevLine{A}{B}
		\RevLine{B}{C}
		\RevLine{C}{D}

		\RevLine{D}{E}
		\RevLine{E}{F}
		\RevLine{F}{G}
		\RevLine{G}{H}

		\RevLine{C}{I}
		\RevLine{I}{J}
		\RevLine{J}{K}

		\RevLine{K}{H}
	\end{psmatrix}


	\caption{A pull of the remote branch will lead to a merge
	because changes in the remote branch does not fast-forward on
	top of the local branch since it contains its own changes.}

	\label{fig:pull-merge}

\end{center}
\end{figure}

An important consideration when pulling or pushing is how to integrate
the changes being synchronized into the repository being updated. The
problem arises because the two repositories may each have diverged since
the last synchronization and it may be necessary to merge in order to
``realign'' changes between the two repositories. This is illustrated
in Figure \ref{fig:pull-merge}. The goal is to ensure that no changes
are lost unintentionally, since there may be scenarios where a
synchronization should overwrite and abandon changes in the repository
being updated. As with merges, fast-forwards are preferable since they
follow the ``append-only'' approach which guarantees that no data is
discarded. For read-only tracking of a remote branch, which is by far
the most common case, each pull will always simply fast-forward the
local branch.

% and is therefore safe in
% that this sort of integration will not result in changes being lost.
% The common case for As with merges, fast-forwards
% follow the preferred

% In a context where the repository being pushed to is shared among
% several developers, the shared repository can diverge between
% synchronizations, so that a  push does no longer fast-forward the
% remote branch. Fast-forward so that the local repository no longer
% reflects what isdeveloper pushing

While the pull operation can be resolved locally by a merge, the same
is not true for the push operation. It is the same problem; however,
when pushing to remote repositories, it is not possible to perform
various operations since the remote repository might be a bare one
without a working tree, merging is impractical. For this reason,
pushing is usually required to always fast-forward, since this
guarantees that the operation will always succeed without conflicts.
So in order to push to a remote branch which has diverged from the
local version of the branch, it is necessary to first pull the
changes, perform the merging locally, and finally push the resulting branch
revision.
